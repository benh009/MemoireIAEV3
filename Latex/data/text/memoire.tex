\documentclass[12pt,a4paper]{./../../template/memoire-umons}

\usepackage{xspace}
\usepackage{lmodern}
\usepackage[babel=true]{microtype}

\usepackage[latin1]{inputenc}
\usepackage[T1]{fontenc}
\usepackage[francais]{babel}
\usepackage{amssymb,amsmath,amsthm}
%\usepackage{hyperref}% hyperliens dans le PDF, pas pour impression

\usepackage{algorithm}
\usepackage{algpseudocode}
\usepackage{tikz}
\usetikzlibrary{arrows,shapes,positioning}
\usetikzlibrary{decorations.markings}

\usepackage{hyperref}




\title{Algorithmes de localisation probabilistes sur une brique EV3}
\author{Beno\^it \textsc{Hofbauer}}
\date{2014--2015}
%\directeur{Hadrien Melot}
\directeurs{Hadrien M\'elot\\ Pierre Hauweele}
%\codirecteurs{Pierre Hauweele}
\service{Algorithmique}

\discipline{informatiques}

%%%%%%%%%%%%%%%%%%%%%%%%%%%%%%%%%%%%%%%%%%%%%%%%%%%%%%%%%%%%%%%%%%%%%%%%
%% Vos macros
\newcommand{\source}[1]{     \textbf{Source:} {#1} }

%%%%%%%%%%%%%%%%%%%%%%%%%%%%%%%%%%%%%%%%%%%%%%%%%%%%%%%%%%%%%%%%%%%%%%%%

% Compile uniquement certains morceaux sans perdre les références
% automatiques et la table des matières des parties déjà compilées :
%\includeonly{introduction,chapitre1}

\begin{document}
% Éventuellement utiliser l'environnement « preface » pour avoir une
% numérotation des pages en chiffres romains.
\begin{preface}
\begin{figure}[H]
\begin{center}
\includegraphics{../img/logoMemoire.png}
\end{center}
\end{figure}

\section*{Remerciements}
Je tiens � remercier Hadrien M�lot le directeur de ce m�moire et Pierre Hauweele le codirecteur de ce m�moire de m'avoir apport� leur expertise dans le domaine et leur aide pr�cieuse. Pierre Hauweele m'a �galement transmis sa passion de la robotique, ce qui m'a permis de trouver la motivation de faire ces recherches dans le domaine de la robotique qui est un domaine de taille importante et qui comporte de nombreuses difficult�s.  

Je tiens �galement � remercier ma m�re qui m'a aid� � relire et structurer ce document ainsi que l'ensemble des membres de ma famille de m'avoir soutenu et aid� chacun � leur fa�on durant cette p�riode. Je remercie particuli�rement ma soeur qui m'a aid� � r�aliser le logo ci-dessus qui symbolise ce m�moire. 

Et finalement, je remercie Sylvain Kempen, �tudiant en sciences math�matiques � l'UMons, de m'avoir aid� � comprendre les aspects math�matiques les plus complexes de ce m�moire. Sa gentillesse et ses capacit�s de vulgarisation des math�matiques m'ont �t� tr�s utiles tout au long de la r�daction de ce m�moire. 

\begin{figure}[H]
\begin{center}
\includegraphics{../img/robot2.png}
\end{center}
\end{figure}

\newpage
\section*{Mots cl�s}

\begin{itemize}
\item Localisation 
\item Extended Kalman Filter Localization
\item Monte Carlo Localization
\item EV3
\item Robotique
\end{itemize}


\section*{Mots cl�s}
\begin{itemize}
\item Localisation 
\item Extended Kalman filter
\item Monte Carlo localization
\item EV3
\item Robotique

\end{itemize}



\section*{R�sum�}
L'objectif de ce m�moire est de d�velopper des aspects th�oriques de la robotique en y liant une validation pratique. La localisation d'un robot dans son environnement est le principal sujet abord�. Les algorithmes de localisation probabilistes, Extended Kalman Filter (EKF) ainsi que Monte Carlo Localization (MCL) seront �tudi�s en profondeur et compar�s. Toutefois, une multitude de sous-sujets sont li�s et d�coulent de la localisation d'un robot dans son environnement. Les principaux sous-sujets sont la construction de cartes de l'environnement du robot, la recherche de chemin entre un point de d�part et un point final, les syst�mes d'exploitation dans la robotique, la caract�risation des capteurs et des actuateurs, l'analyse d'images... Cet ensemble de sous-sujets est abord� dans ce m�moire pour permettre aux lecteurs de comprendre une partie de l'univers de la robotique et lui donner des pistes et des r�f�rences pour entamer des recherches futures. La compr�hension de l'univers de la robotique est primordiale pour int�grer les algorithmes de localisation en un tout coh�rent. 

Le kit EV3 de Lego est utilis� pour la validation pratique des concepts th�oriques pr�sent�s. Ce kit est compos� d'une brique intelligente, de capteurs, de moteurs ainsi que d'�l�ments de construction qui permettent de r�aliser rapidement la structure d'un robot. Ce kit est combin� avec un smartphone tournant sous Android. Ce smartphone permet d'ajouter de la puissance de calcul et de se servir des capteurs pr�sents sur le smartphone. En effet, les smartphones actuels poss�dent un grand nombre de capteurs � couts r�duits et leur capacit� de calcul est importante. Le principal capteur utilis� est la cam�ra du smartphone. 



\newpage

\section*{Avant-propos}
Les deux mots �~voiture autonome~� sont sur les l�vres de tous les constructeurs automobiles depuis quelques ann�es. Plusieurs prototypes ont �t� r�alis�s depuis les ann�es~1980~\cite{Stentz_1985_1234,Kanade:1986:ALV:324634.325197}. Cependant, ces prototypes �taient limit�s soit par la vitesse du v�hicule soit par les restrictions de l'environnement dans lequel ils �voluaient. L'�volution des technologies li�es aux capteurs et l'augmentation de la puissance des processeurs ont permis une avanc�e consid�rable dans le domaine. Google annonce en octobre 2010~\cite{NYT2010} avoir con�u un syst�me de pilotage automatique pour automobile. Ce prototype est capable de se d�placer dans la circulation automobile sans assistance humaine. � ce jour, de nombreux constructeurs automobiles travaillent sur des voitures autonomes. On peut citer Audi, Toyota~\cite{ToyotaBot}, Nissan~\cite{bbc2013}, ou encore Mercedes-Benz.

La multitude de capteurs qui �quipent ces v�hicules est �videmment primordiale. Toutefois, sans traitements et croisements de ces mesures, il est impossible de d�velopper un v�hicule autonome. Pour cette raison, des algorithmes ont �t� d�velopp�s. Ces algorithmes permettent aux v�hicules de se localiser dans leur environnement pour ensuite �tablir leur parcours. Ces algorithmes doivent prendre en compte que les valeurs des capteurs peuvent �tre entach�es d'erreurs. Ils doivent �galement prendre en compte que les capteurs ne permettent de capter qu'une partie de l'environnement du v�hicule.
Des v�hicules comme la voiture autonome de Google embarquent un grand nombre des capteurs de haute pr�cision. Toutefois, �quiper des voitures de ces capteurs se r�v�le encore tr�s on�reux. Il est donc int�ressant de se demander quelles sont les limites des algorithmes de localisation avec des capteurs � faibles couts. Ces algorithmes sont-ils perfectibles~? Les robots de nettoyage domestique~\cite{futura-sciences} sont des exemples de robots qui ont des capteurs simples et peu on�reux, mais qui doivent se d�placer dans des environnements complexes que sont les habitations. En effet, les meubles, les escaliers, les humains et les animaux domestiques sont autant d'�l�ments qui �voluent dans le m�me environnement que le robot. Ils rendent donc la localisation et les d�placements plus complexes pour le robot. 

Les voitures autonomes ainsi que les robots de nettoyage domestique peuvent �tre au m�me titre qualifi�s de robots. On peut facilement assimiler les voitures autonomes et les robots de nettoyage � la d�finition suivante de ATILF\footnote{ATILF (Analyse et traitement informatique de la langue fran�aise)~:  \href{http://www.atilf.fr/ressources/grand-public/mots-de-la-science/cadres/mots/tlfi/robot.htm}{www.atilf.fr}}: �~Appareil effectuant, gr�ce � un syst�me de commande automatique � base de microprocesseur, une t�che pr�cise pour laquelle il a �t� con�u dans le domaine industriel, scientifique ou domestique~�. Dans ce m�moire, le but est donc de faire d�couvrir l'univers de la robotique avec la localisation d'un robot dans son environnement comme sujet principal. Il a �t� choisi de pr�senter ce m�moire de sorte que le lecteur puisse l'utiliser comme la base principale pour impl�menter ses propres algorithmes de localisation sur le robot de son choix. C'est pourquoi les algorithmes sont d�crits, mais �galement compar�s pour permettre au lecteur de choisir les algorithmes les plus adapt�s aux sp�cificit�s et � l'environnement de son robot. De nombreuses citations permettent �galement aux lecteurs de continuer ses recherches selon la direction dans laquelle il souhaite approfondir ses recherches dans la robotique. On peut donc consid�rer ce m�moire comme un tutoriel qui aide � d�broussailler l'univers �norme que repr�sente la robotique. Il est d�coup� en deux parties distinctes. La premi�re d�crit la th�orie. La seconde met en pratique la th�orie pr�sent�e. Les personnes � l'aise avec cette th�orie pourront donc directement passer � la partie pratique. Cependant pour les novices dans la robotique, il est recommand� de lire l'int�gralit� du m�moire, et ce dans l'ordre d'apparition des chapitres. En effet, des exemples concrets sont donn�s pour aider le lecteur � se familiariser avec les diff�rentes notions abord�es.



Le livre de Sebastian Thrun �~Probabilistic Robotics~�~\cite{Thrun:2005:PR:1121596} qui est une bible dans le domaine de la robotique a �t� une ressource importante pour la r�daction de ce m�moire. Rappelons que Sebastian Thrun est l'ing�nieur principal qui a lanc� le projet �~Google driverless car~�. Le m�moire de Pierre Hauweele~\cite{Hauweele:2013} donne certaines d�monstrations th�oriques plus en profondeur alors que ce m�moire est plus centr� sur les aspects pratiques.

Remarque 1 : ce m�moire est �crit suivant la nouvelle orthographe de la langue fran�aise approuv�e par l'Acad�mie fran�aise~\cite{ortho}. Le lecteur ne doit donc pas �tre surpris de voir, par exemple, �~cout~� �crit sans accent circonflexe. 

Remarque 2 : certains pieds de page pr�sentent des URL qui renvoient � des sites web de r�f�rences. Cependant, dans un souci de pr�sentation de ce m�moire ces liens sont limit�s au nom de domaine du site. La version PDF du m�moire permet d'�tre redirig� vers le lien complet. Il est recommand� de m'envoyer un e-mail � l'adresse Hofbauer92@gmail.com pour vous procurer cette version PDF.

Remarque 3 : ce m�moire est destin� � �tre imprim� recto verso pour diminuer son impact environnemental. L'UMons pr�ne fortement les valeurs de la r�duction de l'impact environnemental des universit�s qui sont de grandes consommatrices de papier. Elle  a m�me re�u un prix pour son engagement \footnote{ UMons diminution de l'impact environnemental : \href{https://portail.umons.ac.be/FR/actualites/Pages/PrixCampusresponsables.aspx}{portail.umons.ac.be}}. L'impression recto verso de ce m�moire est donc en accord avec ces valeurs.



\end{preface}


\tableofcontents
\listoffigures

%\include{introduction}
\part{D�finition formelle }
\chapter{Les cartes}
Les cartes de l'environnement du robot sont des �l�ments importants dans ce m�moire. Ces cartes peuvent �tre g�n�r�es dynamiquement ou au pr�alable avant de les passer en param�tre au robot. Elles permettent de se localiser ou bien de trouver un meilleur chemin entre un point de d�part et un point d'arriv�e. Il y a plusieurs types de cartes ayant chacune leurs caract�ristiques. Les sections suivantes sont destin�es � d�crire les cartes les plus r�pandues qui sont les grilles d'occupation et les cartes compos�es de feature.     
\section{Carte compos� de Feature } 
Cette carte est d�finie � l'aide d'une liste ou d'une sous-liste des objets composant l'environnement du robot. La sous-liste peut comporter des �l�ments importants de l'environnement, comme des rep�res. Elle peut �galement contenir la liste des murs de l'environnement sans prendre en compte les meubles ou les �l�ments dynamiques de l'environnement comme des personnes ou objets mobile au cours du temps. � chaque objet de la carte correspondent ses coordonn�es dans l'environnement. La figure~\ref{GridMap2} repr�sente le parcours d'un drone qui se d�place dans un parc et o� les objets cartographi�s correspondent � la cime des arbres du parc. Il est clair dans cet exemple que la carte ne contient pas tout les objets du parc, mais bien un sous-ensemble des points de rep�re les plus importants. Ce technique permet donc de cartographier des environnements de grande taille et ainsi de ne cartographiant que les �l�ments importants.


\begin{figure}
\begin{center}

\includegraphics[scale=0.7]{./../img/featuremap.png}
\caption{Carte compos� de feature }
\source{ Probabilistic Robotics\cite{Thrun:2005:PR:1121596}}
\label{GridMap2}
\end{center}
\end{figure}   

\section{Les grilles d'occupation} 
Les cartes de type grilles d'occupation d�coupent l'environnement du robot en un ensemble de parcelles de m�me taille. � chaque parcelle une valeur est associ�e. Les valeurs peuvent �tre binaires et d�finir si une parcelle est occup�e ou non, ou bien peuvent �tre une variable qui d�finit la probabilit� d'avoir une parcelle vide ou occup�e. Dans le second cas, un seuil qui d�finit si la parcelle est vide et un seuil qui d�finit si la parcelle est occup�e doivent �tre d�finis pour pouvoir utiliser la carte. Ce type de carte � l'avantage de d�finir la pr�sence d'un objet, mais �galement l'absence d'objet ce qui n'est pas le cas des cartes compos� de feature. La figure ~\ref{GridMap2} correspond � la cartographie du campus de Stanford � l'aide d'une carte d'occupation. La carte est g�n�r�e dynamiquement � l'aide d'un capteur de distance infrarouge qui permet de d�finir les parcelles de la carte qui sont vides ou occup�es. Le probl�me principal de ce type de grille est que leur efficacit� est fortement li�e � la taille de la d�coupe. Plus la d�coupe est importante, et plus cette grille fournie des informations pr�cises, mais � contrario plus les performances de mises � jour et de lecteur sont important. Il est donc important de bien d�finir une d�coupe optimale.    

\begin{figure}
\begin{center}

\includegraphics[scale=0.7]{./../img/occupancymap.png}
\caption{Grille d'occupation }
\source{ Probabilistic Robotics\cite{Thrun:2005:PR:1121596}}
\label{GridMap2}
\end{center}
\end{figure}   


\chapter{Probl�me de localisation}

\section{D�finition du probl�me}


\subsection{Explication du probl�me}
Le probl�me de localisation d'un robot mobile consiste � d�terminer sa position � un instant donn� sur une carte donn�e. Pour atteindre cet objectif, le robot a � sa disposition les mouvements qu'il a r�alis�s, des mesures provenant de ses capteurs ainsi qu'une carte de son environnement. 

Cette situation peut facilement �tre compar�e � un promeneur cherchant sa position dans la nature avec une carte topographique. Cette personne n'a � sa disposition que les observations qu'elle peut r�aliser (sans l'aide d'instrument de localisation comme un GPS). Elle peut se localiser � l'aide des montagnes qui sont des points de rep�re int�ressants. Elle peut �galement essayer de trouver des ressemblances avec le chemin qu'elle parcourt et ce qu'elle peut observer sur la carte. Cependant, il est difficile d'estimer exactement la distance qui s�pare cette personne de la montagne. La distance parcourue par cette personne entre deux points est �galement difficile � estimer sans erreur.  Toutes ces informations sont donc approximatives.
Malgr� ces erreurs d'approximations, gr�ce � la quantit� d'informations accumul�es durant son parcours cette personne a de fortes chances d'�tre de plus en plus certaine de sa position. En effet, en d�but de parcours, cette personne peut supposer �tre � un ensemble d'endroits diff�rents � la suite d'un manque d'informations en sa possession. Par la suite gr�ce aux nouvelles informations elle peut proc�der par �limination pour d�terminer sa position.

Il s'av�re que les robots doivent faire face aux m�mes types de probl�mes pour se localiser. Gr�ce � l'odom�trie, il est possible de d�terminer les mouvements du robot en fonction de la rotation de chacun des moteurs du robot. � l'aide de capteurs tels que les capteurs infrarouges, il est possible de d�terminer la distance entre la position du robot et un objet. Cependant comme pour le promeneur ces informations sont entach�es d'erreurs de pr�cisions. De plus, dans le cas des capteurs de distance infrarouges ou ultrasoniques, les ondes peuvent �tre r�fl�chies de fa�on inattendue selon la forme et la mati�re de la surface de l'objet r�fl�chissant l'onde. Il s'av�re que chacun des capteurs poss�de des probl�mes sp�cifiques aux technologies qu'ils utilisent. Une solution na�ve serait de vouloir acheter des d�tecteurs et des moteurs toujours plus pr�cis. Cependant, le cout des capteurs plus pr�cis est plus important. De plus, des erreurs peuvent �tre impossibles � g�rer � l'aide de mat�riel plus pr�cis. En effet, il peut arriver que les roues du robot n'adh�rent pas parfaitement � la route. Ce qui entraine le glissement des roues et donc bien que le robot reste immobile, les moteurs enregistreront un mouvement. Si le robot �volue dans un monde dynamique il peut arriver qu'un personne, ou autre objet passe devant un capteur or cet �l�ment n'est pas repr�sent�e sur la carte. Ce qui pourrait entrainer l'assimilation de cet �l�ment � un autre �l�ment de la carte. Un autre �l�ment d'erreur li� � la carte est sa pr�cision. La carte fournie au robot n'est pas d'une pr�cision infinie. Les �l�ments cartographi�s peuvent se retrouver l�g�rement � c�t� de la position d�finie sur la carte. Les capteurs ont �galement des limitations physiques, il est par exemple impossible pour une cam�ra de voir � travers les murs, ce qui ne donne au robot qu'une vue partielle de l'environnement dans lequel il �volue. Finalement, une erreur sp�cifique aux robots est le kidnapping. Ce qui correspond � l'arr�t du robot, suivi d'un d�placement du robot. Une fois celui-ci remis en marche, il n'a pas conscience qu'il a �t� chang� de place.

L'ensemble des probl�mes et contraintes pr�sent�es met en �vidence qu'il n'est pas possible d'utiliser les donn�es des capteurs et moteurs sans une analyse et un traitement pr�alable. Il pourrait �tre catastrophique de vouloir int�grer une valeur fortement entach�e d'erreur. Cette seule valeur pourrait conduire � d�finir une localisation compl�tement farfelue d'un robot et ceci avec les risques qui en d�coulent. Les techniques de localisation probabiliste qui sont pr�sent�s dans ce m�moire permettent de pallier � ces valeurs erron�es. Les sections suivantes discutent des solutions apport�es aux diff�rents probl�mes discut�s. 



\subsection{Mod�le de mouvement et d'observation}

L'objectif de la localisation d'un robot est de d�finir sur une carte la position du robot � un instant donn�. Cette position est d�not�e par le vecteur $x_t$. 
Dans la suite de ce m�moire, la position du robot est d�finie par ses coordonn�es dans un espace � deux dimensions ainsi que par son orientation. Les algorithmes pr�sent�s ne se limitent pas � cette situation et peuvent �tre utilis�s pour un espace � un nombre de dimensions sup�rieures. L'�tat des bras des robots industriels~\cite{osha.gov} peut �galement �tre repr�sent� � l'aide de l'angle de chaque articulation du bras. Toutefois, les principes sont plus simples � comprendre dans cette situation et cette situation est souvent suffisante pour les robots mobiles. Formellement, la position d'un robot � l'instant $t$ dans un espace � deux dimensions peut-�tre d�finit par le vecteur :
$$ x_t = \begin{pmatrix} x \\ y \\ \theta \end{pmatrix}$$

o� $x, y$ correspondent aux coordonn�es dans l'espace � deux dimensions et $\theta$ correspond � l'orientation du robot (voir ~\ref{PR2D}). 


\begin{figure}
\begin{center}
\includegraphics[scale=0.7]{./../img/PositionRobot.png}
\caption{Position d'un robot dans un espace � deux dimensions }
\source{Probabilistic Robotics\cite{Thrun:2005:PR:1121596}}
\label{PR2D}
\end{center}

\end{figure}

Pour d�terminer la position du robot dans le temps, il faut prendre en consid�ration les contr�les du robot (not�s : $u_t$) ainsi que les observations effectu�es par le robot (not�es : $z_t$). Les algorithmes qui seront pr�sent�s par suite suivent l'hypoth�se de Markov. C'est-�-dire que l'�tat $x_{t}$ ne d�pend que de l'�tat $x_{t-1}$ ainsi que des contr�les $u_{t}$ et des observations courantes $z_{t}$(voir figure ~\ref{HM}). 


\begin{figure}
\begin{center}
\includegraphics[scale=0.7]{./../img/HypotheseMarkov.png}
\caption{Hypoth�se de Markov}
\source{Probabilistic Robotics\cite{Thrun:2005:PR:1121596}}
\label{HM}
\end{center}
\end{figure}

Formellement, les contr�les peuvent �tre d�finis par le vecteur suivant :
$$u_t =  \begin{pmatrix} d_t \\ \gamma_t \end{pmatrix} $$
o� $d_t$ correspond � la distance parcourue et $\gamma_t$ correspond � l'angle de rotation du robot. L'odom�trie permet de d�terminer les commandes $u_t$. Tandis que les valeurs de $u_t$ permettent de d�finir it�rativement les nouvelles positions � l'aide du mod�le de mouvement suivante : 
$$x_{t}= \begin{pmatrix} x_{t-1}+ d_t \cos (\theta_{t-1} + \gamma_t)\\ y_{t-1} + d_t \sin (\theta_{t-1} + \gamma_t)\\ \theta_{t-1} + \gamma_t \end{pmatrix}$$


Les observations effectu�es par le robot peuvent �tre d�finies par le vecteur suivant :
$$ z_t = \begin{pmatrix} d_t^z \\ \rho_t \end{pmatrix}$$
o� $d_t^z$ correspond � la distance entre le robot et l'�l�ment d�tect� et $\rho_t$ correspond � l'angle form� entre l'orientation du robot et la position de l'�l�ment. 



Mod�le d'observation





Ces �quations ne sont vraies que si les valeurs retourn�es par les moteurs et capteurs �taient 100 \% juste, ce qui n'est �videmment pas le cas. En effet, ces donn�es sont sujettes � des erreurs de mesures. Il est int�ressant de remarquer que si l'odom�trie n'�tait pas sujette � des erreurs de mesures, il ne serait pas utile d'�quiper ses robots de capteurs pour les localiser, l'odom�trie serait suffisante. Afin de prendre en compte les erreurs, nous allons red�finir notre mod�le de mouvement. Une erreur de rotation initial ainsi qu'une erreur sur la distance est ajout� � chaque contr�le $u_t$. Pour cel� $\gamma$ et $d$ sont red�fini par $\hat{\gamma} $ et $\hat{d}$ � l'aide des formules suivantes : 

$$
\hat{\gamma} = \gamma - \epsilon_{\alpha_1 \gamma^2 + \alpha_2 d^2 }  
$$

$$
\hat{d} = d - \epsilon_{\alpha_3 d^2  + \alpha_4 \gamma^2 } 
$$

o� $\alpha_1,\alpha_2,\alpha_3,\alpha_3, $ sont des param�tres � d�terminer et sont sp�cifique � chaque robot et $\epsilon_{b^2}$ correspond � une gaussienne de moyenne nulle et de variance $b^2$. Cette formule montre que plus la distance $d$ et l'angle $\gamma$ sont importants et plus l'erreur risque d'�tre grande. Ce qui correspond bien � la r�alit�.   $\alpha_2d^2$ et $\alpha_4 \gamma^2$ correspondent aux erreurs o� une rotation est consid�r�e par une translation et inversement. � l'aide de ces deux formules, le mod�le de mouvement auquel on a ajout� les erreurs de mesures devient donc : 

$$x_{t}= \begin{pmatrix} x_{t-1} + \hat{d}_t  \cos (\theta_{t-1} + \hat{\gamma}_t) \\ y_{t-1} + \hat{d}_t \sin (\theta_{t-1} + \hat{\gamma}_t)\\ \theta_{t-1} + \hat{\gamma}_t  \end{pmatrix}$$


Il est important de remarquer que le mod�le de mouvement et le mod�le  d'observation pr�sent�s sont des exemples et doivent �videmment �tre adapt�s aux diff�rents robots. 



\subsection{Algorithme de localisation de Markov}
Dans ce m�moire les algorithmes d�velopp�s sont des algorithmes de localisation probabiliste. L'approche probabiliste permet d'int�grer les erreurs de pr�cision des capteurs dans l'algorithme et leur objectif est de d�terminer la fonction de densit� de probabilit� du vecteur al�atoire associ�e X. 
$$E \rightarrow [0;1]: x \mapsto p(X = x)$$

 L'algorithme ~\ref{alg:Markovlocalisation} qui est d�crit en pseudocode correspond � l'algorithme de localisation de Markov qui est � la base de tout les algorithmes de localisation qui sont pr�sent�s dans ce m�moire. Il d�crit la mise � jour de la position $x_{t-1}$ vers l'�tat $x_t$. Il prend en param�tre la croyance de la position pr�c�dente (c'est-�-dire la fonction de densit� de probabilit� du vecteur  de position x), le contr�le courant, les observations courantes ainsi que la carte dans laquelle le robot �volue. Il est constitu� d'une boucle principale, qui it�re sur toutes les valeurs possibles pour la position $x_t$. Cette boucle contient deux �tapes importantes. La premi�re �tape se nomme �la pr�diction� et consiste � calculer une croyance temporaire $\overline{bel}$ de la position du robot � l'aide de $u_t$ et de la croyance de l'�tape pr�c�dente $bel(x_{t-1})$. La seconde �tape correspond � la mise � jour de la croyance $bel(x_t)$ � l'aide des mesures $z_t$ et de la croyance $\overline{bel}(x_t)$ calcul�e dans l'�tape de pr�diction.

\begin{algorithm}
\caption{ Localisation de Markov  }\label{alg:Markovlocalisation}
\begin{algorithmic}[1]
\Procedure{Markov}{$bel(x_{t-1}),u_t , z_t, m $}
\ForAll {$ x_t $}
\State $\overline{bel}(x_t) \gets \int p(x_t \mid u_t, x_{t-1},m)bel(x_{t-1}) dx_{t-1} $  \Comment{pr�diction}
\State $bel(x_t) \gets \eta  p(z_t \mid x_t, m )\overline{bel}(x_t)$  \Comment{mise � jour}
\EndFor
\State \textbf{return} $bel(x_t)$
\EndProcedure
\end{algorithmic}
\end{algorithm}

La figure ~\ref{ILM} illustre une situation o� l'algorithme de Markov est appliqu�. Dans cette illustration, le robot se d�place dans un monde en 1 dimension. Le robot est capable de se d�placer vers la droite ou la gauche. Il peut d�terminer avec une certaine probabilit� s'il se trouve devant une porte ou non. Il peut aussi d�terminer avec une certaine probabilit� la position dans laquelle il se trouve � l'aide des d�placements qu'il a effectu�s. Dans l'image � a �, le robot n'a encore effectu� aucun d�placement ni observation et n'a aucune information initiale sur sa position. Il a donc une probabilit� uniforme de se trouver sur n'importe quel point de la carte. Dans l'image � b �, le robot observe qu'il se trouve devant une porte. Cette observation permet au robot de d�duire qu'il est devant une des trois portes de la carte. La probabilit� autour des portes augmente en cons�quence. Dans l'image � c �, le robot se d�place vers la droite. Ce qui implique de d�placer �galement la fonction de la croyance de sa position initiale. Ce d�placement implique une diminution de la croyance de sa position due aux erreurs d'estimation du d�placement. Cette diminution de la croyance est repr�sent�e par un aplatissement des diff�rentes gaussiennes. Dans l'image � d �, le robot d�couvre � nouveau une porte. Ce qui augmente encore sa croyance en sa position. Et finalement, l'image � e �, d�montre encore une fois que les d�placements diminuent la croyance de la position du robot. 

\begin{figure}
\begin{center}
\includegraphics[scale=0.7]{./../img/LocalisationMarkovExemple.png}
\caption{Id�e g�n�rale de la localisation de Markov}
\source{Probabilistic Robotics\cite{Thrun:2005:PR:1121596}}
\label{ILM}
\end{center}
\end{figure}


La pr�dictibilit� de l'environnement est un �l�ment important dans le choix d'appliquer ou non des algorithmes probabilistes. Dans le cas d'un environnement bien structur� comme une chaine de montage, le degr� d'impr�dictibilit� est bien moins important que lorsque le robot �volue en ville ou dans une maison. En effet, l'environnement d'une chaine de montage est beaucoup moins sujet � des �l�ments impr�vus comme des personnes ou objets inconnus venant s'ajouter � son environnement de travail. Par la nature impr�dictible des �tres vivants, l'impr�dictibilit� de l'environnement augmente fortement lorsque le robot �volue dans un environnement en pr�sence d'�tre vivant. Les algorithmes probabilistes permettent de pallier � l'impr�dictibilit� d'un environnement. Lorsque l'environnement est forte impr�dictible, il est souvent plus prudent d'augmenter le nombre de capteurs. Des capteurs de proximit� d'un robot de chaine de montage peuvent �tre ajout�s pour �viter un accident si une personne rentre dans le champ de mouvement du robot. Cependant, dans le cas des robots de chaines de montage la vitesse d'ex�cution du robot est tr�s importante pour la productivit� de l'entreprise. Il est donc pr�f�rable de ne pas diminuer cette vitesse avec des v�rifications de s�curit�s et plut�t d'interdire l'acc�s aux alentours de la zone de travail du robot.   

Les algorithmes probabilistes souffrent de deux d�fauts importants. La premi�re est la complexit� en temps de calcul de l'algorithme qui augmente. En effet, dans les algorithmes probabilistes on consid�re toute la fonction de densit� de probabilit� lorsque les algorithmes classiques (non probabiliste) ne consid�rent qu'un �l�ment. La deuxi�me est le besoin d'utiliser des approximations de la densit� de probabilit� exacte. Consid�rer la fonction de densit� exacte devient vite impossible � calculer et est donc indispensable d'utiliser des approximations comme des Gausiennes ou un nombre restreint des �l�ments de la fonction de densit�. Dans certaines situations, ces repr�sentations peuvent �tre �loign�es de la r�alit�. Cependant, l'augmentation de la puissance de calcul des processeurs ainsi que les recherches d'algorithmes plus efficients permettent de grandes �volutions dans le domaine. Toutefois, ces deux points restent encore probl�matiques. Ces deux points sont donc discut�s dans la pr�sentation des algorithmes de localisation suivants.   


\section{Algorithmes de r�solution du probl�me de localisation}
L'algorithme de Markov permet de donner l'id�e g�n�rale des algorithmes de localisation. Cependant pour pouvoir impl�menter concr�tement un algorithme de localisation un certain nombre de questions sont encore ouvertes. La plus importante correspond � la repr�sentation de la fonction de probabilit�. Deux grandes approches existent. La premi�re d�finit une fonction de probabilit� � l'aide de ses param�tres. Et la seconde repr�sente la probabilit� � l'aide d'un certain nombre d'�l�ments discrets. Les sections suivantes discutent de ces deux approches. Elles pr�sentent les algorithmes EKF et MCL qui font partie des algorithmes de localisation les plus connus et qui pr�sentent de bons r�sultats en pratique. 

\subsection{ Algorithme param�trique(EKF)}
Les algorithmes param�triques permettent de repr�senter les croyances de la position d'un robot � l'aide de lois de probabilit�. Les concepts math�matiques de l'algorithme de Kalman ont �t� d�velopp�s dans les ann�es 60 \cite{Kalman61newresults}. Dans le cas de l'algorithme Kalman Filter(KF) la loi de probabilit� est une loi normale. Elle d�pend donc de deux param�tres son esp�rance $\mu$ et son �cart type $\sigma $. Cette loi normale est une loi normale multivari�e lorsque le vecteur de position est compos� de plusieurs �l�ments. On d�finit alors la moyenne de cette fonction multivari�e comme suit  $$\mu = x_t =  \begin{pmatrix} x\\y\\ \theta  \end{pmatrix}$$ 


\begin{algorithm}
\caption{ Kalman filter  }\label{alg:kalmanFilter}
\begin{algorithmic}[1]
\Procedure{KalmanFilter}{$ \mu_{t-1}, \Sigma_{t-1}, u_t, z_t  $}  
\State $ \overline{\mu_t} \gets  A_t \mu_{t-1} + B_t u_t$  \Comment{pr�diction}
\State $ \overline{\Sigma_t } \gets A_t + \Sigma_{t-1} A_t^T+ R_t$ \Comment{pr�diction}
\State $ K_t \gets \overline{\Sigma}_t C_t^T (C_t \overline{\Sigma}_t C^T_t + Q_t )^{-1}$ \Comment{Kalman Gain}
\State $ \mu_t \gets  \overline{\mu}_t  + K_t(z_t - C_t \overline{\mu}_t)  $ \Comment{mise � jour}
\State $ \Sigma_t \gets (I - K_tC_t)\overline{\Sigma}_t$ \Comment{mise � jour}
\State \textbf{return} $ \mu_t,\Sigma_t$
\EndProcedure
\end{algorithmic}
\end{algorithm}

\begin{algorithm}
\caption{ Extended Kalman filter  }\label{alg:ExtendedKalmanFilter}
\begin{algorithmic}[1]
\Procedure{ExtendedKalmanFilter }{$ \mu_{t-1}, \Sigma_{t-1}, u_t, z_t  $}  
\State $ \overline{\mu_t} \gets  g(u_t,\mu_{t-1})$  \Comment{pr�diction}
\State $ \overline{\Sigma_t } \gets G_t \Sigma_{t-1} G_t^T+ R_t$ \Comment{pr�diction}
\State $ K_t \gets \overline{\Sigma}_t H_t^T (H_t \overline{\Sigma}_t H^T_t + Q_t )^{-1}$ \Comment{Kalman Gain}
\State $ \mu_t \gets  \overline{\mu}_t  + K_t(z_t - h(\overline{\mu}_t))  $ \Comment{mise � jour}
\State $ \Sigma_t \gets (I - K_tH_t)\overline{\Sigma}_t$ \Comment{mise � jour}
\State \textbf{return} $ \mu_t,\Sigma_t$
\EndProcedure
\end{algorithmic}
\end{algorithm}


L'algorithme Extended Kalman filter(EKF) correspond � la version non linearaire de l'algorithme du filtre de Kalman. Cette variante a �t� d�velopp�e 
quelques ann�es plus tard  par la NASA\cite{Smith1962} pour faire face au fait que la plupart des syst�mes physiques ne sont pas lin�aires. Pour ce faire les fonctions $g$ et $h$ ont �t� introduites. Ces fonctions ne doivent pas obligatoirement �tre lin�aire mais doivent �tre d�rivables. Contrairement ou filtre de Kalman classique o� les fonctions sont obligatoirement lin�aire pour pr�server des fonctions de r�partition gausienne.



\subsection{Algorithme non param�trique(MCL) }
� l'inverse des algorithmes param�triques, les algorithmes non param�triques ne sont pas bas�s sur une loi de probabilit� connue dont l'algorithme arrange les param�tres pour correspondre au mieux � la croyance de la position. Dans les algorithmes non param�triques, la croyance est repr�sent�e par nombre d�termin� de positions suppos�es. Une probabilit� est associ�e � ces positions suppos�es. Plusieurs techniques existent pour repr�senter ses positions suppos�es. 

La premi�re technique consiste � d�couper la carte de l'environnement en une grille o� chaque �l�ment de la grille correspond � une position suppos�e\cite{4621305}. Pour repr�senter l'orientation du robot, il faut multiplier le nombre de cases par le nombre d'angles d'orientation que peut prendre le robot (voir ~\ref{img:gridMap}).Dans cette repr�sentation, seules trois orientations sont possibles. Ces trois orientations correspondent aux trois plans de la repr�sentation. Comme on peut s'en rendre compte, il est tr�s important de d�finir la bonne granularit� de la d�coupe. Une d�coupe trop importante augmente le temps de calcul, alors qu'une grille trop peu d�coup�e rend la localisation trop peu pr�cise. Dans ce type de d�coupe, plus la carte est grande et plus le temps de calcul est important. 

Dans le cas de l'algorithme de Monte Carlo localization (MCL) aussi appel� Particle filter localization \cite{bib:Rekleitis2004}  car il utilise un filtre � particule, une approche diff�rente a �t� choisie. Dans cet algorithme (voir l'algorithme ~\ref{alg:MCL }) la croyance de la position est repr�sent�e par M particules(voir ~\ref{img:MCL}). Chaque particule est consid�r�e comme une hypoth�se sur la position du robot. Plus une r�gion de la carte contient de particules et plus la probabilit� que le robot s'y trouve est grande. Contrairement aux algorithmes bas�s sur la d�coupe de la carte en une grille, MCL n'implique pas un temps de calcul suppl�mentaire lorsque la taille de la carte augmente. Cependant, il est possible de choisir d'augmenter le nombre de particules pour augmenter la pr�cision. 

MCL souffre d'un probl�me important, en particulier quand $M< 50$ et que l'environnement du robot est grand. Il peut arriver que l'ensemble des particules converge vers une position erron�e. Une fois cette convergence atteinte il est difficile d'en sortir. Pour pallier � ce probl�me, � chaque it�ration un certain nombre de particules sont redistribu�es al�atoirement dans la carte. 

\begin{figure}
\begin{center}
\includegraphics[scale=0.6]{./../img/mcl.png}
\caption{Illustration de MCL, les traits gris correspondent aux particules et le niveau de rouge repr�sente la probabilit� associ�e}
\source{\href{https://en.wikipedia.org/wiki/Monte_Carlo_localization}{Wikipedia}, Auteur : Daniel Lu }
\end{center}

\label{img:MCL}
\end{figure}


\begin{figure}
\begin{center}
\includegraphics[scale=0.6]{./../img/gridMap.png}
\caption{Illustration de la carte en grille}
\source{Probabilistic Robotics\cite{Thrun:2005:PR:1121596}}
\end{center}
\label{img:gridMap}
\end{figure}



\begin{algorithm}
\caption{ MCL  }\label{alg:MCL }
\begin{algorithmic}[1]
\Procedure{MCL }{$ \mathcal{X}_{t-1}, u_t, z_t ,m $}  
\State $ \overline{\mathcal{X}_t} \gets  \emptyset $  
\State $ \mathcal{X}_t \gets  \emptyset $  

\For {$ m = 1$ to  M }
\State  $ x^{[m]}_t \gets sample\_motion\_model(u_t , x^{[m]}_{t-1})$
\State $w_t^{[m]} \gets measurement\_model(z_t , x_t^{[m]},m)$
\State $ \overline{\mathcal{X}}_t \gets \overline{\mathcal{X}}_t + \langle  x^{[m]}_t , w_t^{[m]} \rangle $
\EndFor

\For {$ m = 1$ to  M }
\State draw i  with probability $\propto w^{[i]}_t$
\State add $x^{[i]}_t$ to $\mathcal{X}_t$
\EndFor
\State \textbf{return} $ \mathcal{X}_t$
\EndProcedure
\end{algorithmic}
\end{algorithm}


\subsection{Comparaison de MCL et EKF}
Le tableau ~\ref{MCLVSEKF} donne et compare les principales caract�ristiques de l'algorithme EKF et MCL. Ce tableau comparatif permet de mettre en valeur qu'aucun algorithme n'est globalement meilleur. Ils poss�dent chacun leurs qualit�s et d�fauts. Les d�fauts de l'un se r�v�lent les qualit�s de l'autre. Par exemple, EKF est efficient en temps et m�moire contrairement au MCL dont l'efficience d�pend fortement du nombre de particules. 

Cependant, MCL est plus robuste � EKF. En effet, la repr�sentation de la fonction de probabilit� � l'aide d'une loi normale permet � EKF d'�tre efficient. Cependant, le revers de cette efficience est qu'il est moins robuste lorsque la fonction de probabilit� est fortement diff�rente d'une loi normale. Prenons l'exemple d'un long couloir avec un grand nombre de portes et o� le robot n'est pas capable de distinguer les portes. Dans cette situation MCL peut donner de meilleures performances. En effet, EKF assume que la croyance de la position est proche d'une distribution gaussienne et a donc de mauvaises performances lorsque la croyance correspond plut�t � distribution multimodale. Pour pallier � ce probl�me, l'algorithme classique EKF a �t� am�lior�. Multi-hypothesis tracking (MHT)\cite{1263228} filtre permet de repr�senter la croyance � l'aide d'un mixte de plusieurs gaussienne et donc d'avoir un algorithme plus robuste et efficient.

La localisation globale correspond � un probl�me de localisation o� la position initiale n'est pas connue et l'incertitude est donc grande � cet instant. Il s'av�re qu' EKF est plus appropri� pour suivre la position d'un robot dont on connait d�j� la position initiale. En effet, une repr�sentation unimodale est g�n�ralement une bonne repr�sentation dans un probl�me o� il s'agit de suivre une position, mais pas dans un probl�me de localisation globale. La lin�arisation dans EKF ne fait qu'accroitre ce probl�me en risquant de converger vers une mauvaise position.


De plus, MCL permet de traiter directement dans l'algorithme des mesures brutes or EKF n�cessite des rep�res. Il est donc possible � l'aide de MCL d'utiliser directement les valeurs de capteurs de distance entre le robot et des murs pour les comparer avec une carte repr�sentant les murs. Ce qui n'est pas possible � l'aide de EKF qui n�cessite une carte compos�e d'un nombre limit� de rep�res qui permet de localiser le robot � l'aide des rep�res mesur�s dans son environnement. Finalement, en pratique il s'av�re que MCL est plus simple � impl�menter qu'EKF.


\begin{table}
\begin{center} 

\begin{tabular}{l | c | c }
               & EKF & MCL \\
               \hline
Mesures & Rep�res & Brute \\ 
Erreur de Mesure & Gaussienne & Toute \\
Posterior &Gaussienne & Particules \\
Efficience(m�moire) & ++ & + \\
Efficience(temps)& ++ & + \\
Facilit� d'impl�mentation & + & ++ \\
R�solution & ++ & + \\
Robuste & - & ++ \\ 
Localisation Globale & non & oui\\ 
\hline 
\end{tabular}
\caption{Comparaison EKF et MCL}
\label{MCLVSEKF} 
\end{center}
\end{table}


\chapter{Simultaneous Localization And Mapping}
Les algorithmes de type Slam (Simultaneous Localization And Mapping) sont l'�tape suivante de l'ind�pendance des robots. En effet, dans les algorithmes de simple localisation, la carte de l'environnement doit �tre construite avant de la passer en param�tre aux algorithmes de localisation. Comme son nom l'indique, dans un algorithme Slam la carte est construite en parall�le avec la localisation du robot. 


\section{EKF SLAM}
EKF SLAM est un algorithme SLAM et comme sont nom le laisser penser est bas� sur l'algorithme de localisation EKF. Il utilise �galement une gaussienne pour repr�senter son �tat. Cependant l'�tat contient � l'instant $t$ en plus de la pose du robot, la carte qui est repr�sent�es comme l'ensemble des poses des features d�couvert � l'instant $t$. Cet algorithme est aussi d�coup� en deux �tapes important, l'�tape de pr�diction ainsi que l'�tat de correction. Dans l'�tat de pr�diction la fonction de mouvement n'influence pas les valeurs de la moyenne des poses des features. Ce qui est normal car les features ne bouge par m�me si le robot change de position.   

\begin{algorithm}
\caption{ EKFSLAM  }\label{alg:EKFSLAM }
\begin{algorithmic}[1]
\Procedure{EKFSLAM }{$ \mu_{t-1}, \Sigma_{t-1},  u_t , z_t, m,c_t  $}  
\State $ F_x \gets 
\begin{pmatrix}
1&0& 0\\
0&1&0\\
0&0&1\\
\end{pmatrix}
$


\State $\overline{\mu}_t \gets \mu_{t-1} +  F^T_x
\begin{pmatrix}
d \cos \\
d \sin \\
\gamma \\
\end{pmatrix}
$

\State $G_t \gets I + F_x^T 
\begin{pmatrix}
0,0\\
0,0\\
0,0\\
\end{pmatrix}$


\State $\overline{\Sigma}_t \gets G_t \Sigma_{t-1}G_t^T + F_x^TR_tF_x $

\State $ Q_t \gets 
\begin{pmatrix}
\sigma^2_r&0&0\\
0&\sigma^2_r&0\\
0&0&\sigma^2_r\\
\end{pmatrix}$

\ForAll{ observed features   {$ z^i_t \gets (d^i_t,\rho^i_t)^T $ }}
\State $j \gets c_t^i$
\If{landmark j never seen before}
\State $
\begin{pmatrix}
\overline{\mu}_{j,x}\\
\overline{\mu}_{j,y}\\
\end{pmatrix}
\gets 
$ 
\EndIf
 
\State $q \gets (m_{j,x}-\overline{\mu}_{t,x} )^2 + (m_{j,y}-\overline{\mu}_{t,y})^2$
\State $ \hat{z}^i_t \gets 
\begin{pmatrix}
\sqrt{q}\\
atan2(m_{j,y}-\overline{\mu}_{t,y},m_{j,x}-\overline{\mu}_{t,x} )- \overline{\mu_{t,\theta}}\\
\end{pmatrix}
$
\State $H^i_t \gets
\begin{pmatrix}
-\frac{m_{j,x}-\overline{\mu}_{t,x}}{\sqrt{q}}     &    -\frac{m_{j,y}-\overline{\mu}_{t,y}}{\sqrt{q}}   &    0\\
\frac{m_{j,y}-\overline{\mu}_{t,y}}{q} & -\frac{m_{j,x}-\overline{\mu}_{t,x}}{q}            &  -1\\

\end{pmatrix}
$ 

\State $S^i_t \gets H^i_t \overline{\Sigma_t} [H^i_t]^T + Q_t $ 
\State $ K_t^i \gets \overline{\Sigma}_t [H_t^i]^T [S^i_t ]^{-1}$ \Comment{Kalman Gain}

\State $ \overline{\mu}_t \gets  \overline{\mu}_t  + K_t^i(z_t^i - \hat{z}^i_t ))  $ \Comment{mise � jour}
\State $ \overline{\Sigma}_t \gets (I - K_t^iH_t^i)\overline{\Sigma}_t$ \Comment{mise � jour}

\EndFor

\State $ \mu_t \gets  \overline{\mu}_t $  

\State $ \Sigma_t \gets \overline{\Sigma}_t $  

\State \textbf{return} $ \mu_t , \Sigma_t $
\EndProcedure
\end{algorithmic}
\end{algorithm}

\chapter{Algorithmes de recherche du meilleur chemin}
Les algorithmes de recherche de meilleur chemin consistent � d�terminer le chemin entre un point de d�part et un point d'arriv�e qui sont d�finis sur la carte. Bien entendu, il est souhaitable que ce chemin ne conduise pas litt�ralement le robot droit dans le mur ou ne le fasse pas tomber dans les escaliers dans le cas d'un robot aspirateur domestique. Ce chemin est d�fini par un ensemble de points que doit suivre le robot pour se d�placer du point de d�part jusqu'au point d'arriv�e. Dans la figure ~\ref{CHnonValide} les obstacles sont repr�sent�s par la couleur gris�e et les positions libres qui permettent au robot de se d�placer sans percuter d'objet sont repr�sent�es en blanc. Il est donc �vident que le chemin repr�sent� n'est pas souhaitable, car celui-ci risque de faire percuter le robot avec un objet de son environnement. Les sections suivantes d�crivent comment construire un graphe � l'aide d'une telle carte. Une fois ce graphe construit, il est possible d'appliquer les algorithmes classiques de recherche de chemin dans un graphe. Les plus connus sont A star et Dijksta. Il est donc possible d'utiliser les connaissances disponibles dans le domaine de la th�orie des graphes qui est un domaine relativement bien connu.

\begin{figure}
\begin{center}

\includegraphics[scale=0.7]{./../img/invalid_path.png}
\caption{Chemin non valide }
\source{\href{https://en.wikipedia.org/wiki/Motion_planning}{Wikipedia}, Auteur : Simeon87 }
\label{CHnonValide}
\end{center}
\end{figure}






\section{Graphe}
Pour pouvoir d�terminer un chemin qui ne risque pas de percuter les objets de l'environnement, ses algorithmes n�cessitent de poss�der une ou plusieurs cartes de l'environnement du robot. Ces cartes peuvent avoir �t� g�n�r�es dynamiquement par le robot ou bien construites au pr�alable. Elles serviront � produire un graphe o� les noeuds repr�sentent des positions possibles du robot et les arr�tent les chemins qui permettent de rejoindre ces diff�rentes positions. La figure ~\ref{CHValide} pr�sente le graphe construit � l'aide de la carte de la figure  ~\ref{CHnonValide} . Pour construire ce graphe � partir de cette carte, une technique est de d�couper l'environnement du robot en une carte grillag�e o� chaque �l�ment de la grille prend une valeur occup�e ou libre selon qu'un objet se trouve � l'int�rieure ou non. Les ar�tes du graphe repr�sentent le lien entre deux cases adjacentes libres dans la grille. Un cout d'une unit� est associ� � ces ar�tes. Le cout correspond � la distance entre chaque noeud. Si une case n'est pas libre, aucun lien ne la relit. Ce qui repr�sente qu'il ne faut pas passer par cette case pour d�finir le chemin du robot. Il est ainsi possible de passer de case en case pour d�terminer le chemin entre un point de d�part et un point d'arriv�e. Le cout total du chemin correspond � la somme des couts des ar�tes emprunt�es pour aller du point de d�part jusqu'au point d'arriv�e. Il est �galement possible de d�terminer si un point d'arriv�e n'est pas joignable. Ce qui se produit lorsqu'il est impossible d'y acc�der sans percuter des objets de l'environnement.  

\begin{figure}
\begin{center}

\includegraphics[scale=0.7]{./../img/map_path.png}
\caption{Chemin valide construit � l'aide d'un graphe }
\source{\href{https://en.wikipedia.org/wiki/Motion_planning}{Wikipedia}, Auteur : Simeon87 }
\label{CHValide}
\end{center}
\end{figure}

Plus la grille � un d�coupage important, et plus le chemin retourn� est pr�cis. Cependant, un d�coupage plus important augmente le temps de calcul de fa�on exponentielle. En plus d'un param�tre qui permet de d�terminer la granularit� de la carte, un param�tre qui permet de d�finir la distance � laquelle le robot peut �tre proche des murs doit �tre d�fini. Cette valeur permet de prendre en consid�ration la largeur du robot ainsi que ses capacit�s de rotations. En effet, il est commun que la position du robot d�fini le centre du robot, cependant celui-ci poss�de une largeur qu'il faut prendre en consid�ration pour �viter de percuter les murs avec les c�t�s du robot. 
 

\section{Dijkstra}
L'algorithme de Dijkstra publi� en 1959 par son inventeur du m�me nom permet de r�soudre le probl�me du plus court chemin en une complexit� polynomiale en th�orie des graphes. Dans le cas de la recherche d'un chemin valide pour un robot, cet algorithme prend en param�tre le graphe d�fini dans la section pr�c�dente et retourne le plus court chemin valide. Une condition suppl�mentaire sur le graphe est n�cessaire pour appliquer Dijkstra. Le graphe doit �tre connexe pour trouver le plus court chemin, c'est-�-dire qu? entre chaque sommet du graphe il doit exister un chemin. Dans le cas contraire, il est �videmment impossible de d�terminer un plus court chemin entre deux positions non joignables l'une � l'autre. L'algorithme ~\ref{alg:Dijkstra}  correspond au pseudocode de l'algorithme de Dijkstra. Il est compos� de deux parties principales. La premi�re permet de d�finir la distance entre le noeud de d�part et l'ensemble des noeuds du graphe. L'algorithme proc�de de fa�on it�rative pour d�finir successivement les distances entre le noeud de d�part et les noeuds les plus proches. La seconde partie permet de reconstruire le chemin entre les deux noeuds en partant du noeud de fin jusqu'� arriver au noeud de d�part. L'algorithme de Dijksta est robuste et trouve toujours le meilleur chemin. Cependant, lorsque la taille du graphe devient importante et qu'on souhaite trouver rapidement le plus court chemin il est int�ressant d'utiliser l'algorithme A star qui est en moyenne plus rapide que Dijkstra. Cependant dans le pire des cas ils ont une complexit� identique. La section suivante d�crit donc l'algorithme A star. 


\begin{algorithm}
\caption{ Dijkstra  }\label{alg:Dijkstra}
\begin{algorithmic}[1]
\Procedure{Dijkstra }{$G = (S,A), S_{deb} ,S_{fin}$}  
\State Initialiser tous les sommets comme non marqu� et donn� un valeur $+ \infty$ � tous les labels L 
\State $L(S_{deb}) \gets 0$
\While{Il existe un sommet non marqu�  }

\State choisir et marquer le sommet $a$ non marqu� de plus petit label L
\ForAll{sommet b non marqu� voisin de $a$ }
\If{$L(b)> L(a)+v(a,b)$}
\State $ L(b) \gets  L(a)+v(a,b) $ 
\State $b.pr�d�dent \gets a $
\EndIf

\EndFor
\EndWhile
\State $S_n \gets S_{fin}$ 
\While{$S_n != S_{deb}$}
\State $chemin \gets chemin + S_n$ 
\State $S_n \gets S_n.pr�c�dent $
\EndWhile


\State \Return $ chemin$
\EndProcedure
\end{algorithmic}
\end{algorithm}


\section{A star}
L'algorithme A star a �t� publi� en 1968  et fonctionne de fa�on relativement semblable � l'algorithme de Dijkstra, mais celui-ci permet d'obtenir de meilleure performance que Dijksta gr�ce � l'utilisation d'une heuristique. Cette heuristique donne une estimation du cout jusqu'aux destinations recherch�es. Pour appliquer A star son heuristique doit �tre admissible c'est-�-dire que la fonction heuristique ne doit jamais surestimer la valeur r�elle du cout. Dans la recherche du plus court chemin valide de notre robot, l'heuristique repr�sente la distance � vole d'oiseau qui est toujours plus courte ou �gale � n'importe quel chemin. 


\begin{algorithm}
\caption{ A star  }\label{alg:AStar}
\begin{algorithmic}[1]
\Procedure{A star }{$G = (S,A), S_{start} ,S_{goal}$}  

\State $closedSet \gets \emptyset $   \Comment{les noeuds d�j� �valu�s}
\State $openSet \gets \{ S_{start}\}$
\State $came\_from \gets \emptyset$

\State $g\_score \gets +\infty  $  \Comment{$+\infty$ pour tous les �l�ments}
\State $g\_score[S_{start} ] \gets 0 $
\State $  f\_score \gets  +\infty$
\State $f \_score[S_{start} ] \gets g\_score[S_{start} ]+ h(S_{start} ,S_{goal})$
\While{openset !=$\emptyset$}
\State $S_{current} \gets $node with lowest $f\_score[]$
\If{$S_{current} == S_{goal}$ }
\State \Return $reconstruct\_path(came\_from,S_{goal})$ 
\EndIf  
\State $openSet \gets openSet - S_{current} $
\State $closedSet \gets closedSet + S_{current}$
\ForAll{ $S_{neighbor} \in$ neighbor\_nodes($S_{current}$)  $ \& \notin closedSet$ }
\State $t\_g\_score \gets g[S_{current}]+ dist(S_{current},S_{neighbor})$
\If{$S_{neighbor } \notin openSet || t\_g\_score < g\_score[S_{neighbor}]$}
\State $came\_from[S_{neighbor}] \gets S_{current}$
\State $g\_score[S_{neighbor}] \gets t\_g\_score$
\State $f\_score[S_{neighbor}] \gets  g\_score[S_{neighbor}]+ h(S_{neighbor},S_{goal})$
\If {$S_{neighbor} \notin openSet $}
\State $openSet \gets openSet + S_{neighbor}  $
\EndIf
\EndIf
\EndFor
\EndWhile
\State \Return $failure$
\EndProcedure
\Procedure{$reconstruct\_path$}{$came\_from,S_{current}$}
\State $path \gets \{S_{curentl}\} $ 
\While{$S_{current} \in came\_from$}
\State $S_{current} \gets came\_from[S_{current}]$ 

\State $path \gets path + S_{current}$

\EndWhile


\State \Return $ path$
\EndProcedure
\end{algorithmic}
\end{algorithm}






\part{Cas pratique }
\chapter{Lego Mindstorms}
Les Legos sont des jouets de construction fabriqu�s par le groupe danois �the Lego Group�. La s�rie Mindstorms correspond � la gamme � robotique programmable � de Legos. Cette s�rie est vendue en kits qui permettent de r�aliser certains mod�les de robots pr�d�finis et par la suite de laisser cours � son imagination. Les kits contiennent une brique intelligente programmable, un ensemble de capteurs et de moteurs ainsi qu'un ensemble d'�l�ments de constructions qui proviennent de la gamme Lego Technic. Ces kits permettent de r�aliser rapidement et � couts mod�r�s des robots simples. Ils se sont donc vite r�v�l�s comme des outils int�ressants pour l'apprentissage de la robotique et de la programmation.

\section{Description du hardware  }

\subsection{Brique intelligente}
La brique intelligente programmable (voir Figure ~\ref{EV3} ) est le cerveau des robots EV3. Elle est dot�e d'une interface � six boutons lumineux qui changent de couleur pour indiquer l'�tat d'activit� de la brique, d'un affichage � haute r�solution noir et blanc, d'un hautparleur int�gr�, d'un port USB, d'un lecteur de cartes mini SD, de quatre ports d'entr�e et de quatre ports de sortie. La brique prend �galement en charge la communication USB, Bluetooth et Wifi avec un ordinateur. Son interface programmable permet � la fois de programmer et de journaliser des donn�es directement sur la brique. Elle est compatible avec les appareils mobiles et est aliment�e par des piles AA ou par la batterie CC rechargeable EV3. La brique EV3 fournit l'�nergie pour les capteurs et les moteurs. Elle permet de r�cup�rer et de traiter les informations des capteurs et d'envoyer des commandes aux moteurs par le biais des ports d'entr�e et de sortie. Ce qui rend la connexion et l'utilisation des capteurs et des moteurs tr�s facile.

\begin{figure}
\begin{center}
\includegraphics[scale = 0.7]{./../img/EV3.png}
\end{center}
\caption{Brique EV3}
\label{EV3}
\end{figure}


Depuis son lancement la gamme de Lego Mindstorms a connu trois types de briques qui se sont succ�d�, la brique RXC, NXT et EV3. Elles ont progressivement augment� de puissance de calcul. La vraie r�volution de la brique EV3 est la possibilit� de booter le syst�me sur une carte SD. Ce qui augmente ainsi le potentiel de la brique (comme d�crit dans la section ~\ref{sec:DesSoftware}). 

La brique EV3 embarque un processeur arm9, 16Mo de m�moire flash et 64Mo de RAM. La carte SD permet d'�tendre la m�moire � 32Go. Il est vrai que la puissance a augment� dans la brique EV3 en comparaison avec ses pr�d�cesseurs. Toutefois, cette puissance de calcul reste r�duite en comparaison aux ordinateurs et smartphones actuels. 
\subsection{Moteurs et capteurs}
Le pack de base de la bique EV3 est compos� de deux moteurs moyens, un petit moteur, un capteur de pression, un capteur de distance infrarouge et d'un capteur de couleurs. Il est possible d'acheter des capteurs Legos suppl�mentaires tels qu'un gyroscope, un capteur de distance ultrasonique et une boussole. Il est �galement possible d'acheter d'autre type de capteurs d�velopp�s par des soci�t�s ind�pendantes de Lego. C'est possible gr�ce au support de protocole standard que doivent adopter les capteurs de donn�es. On peut par exemple citer $I^2C$ qui est un bus de donn�es con�u par Philips et qui est tr�s r�pandu dans le monde de l'�lectronique. On voit donc en vente des capteurs GPS, des capteurs RFID, des capteurs d'humidit� qui sont compatible avec la brique EV3

La partie qui suit d�crit plus en profondeur les capteurs et moteurs du kit de base qui sont ceux disponibles pour r�aliser ce projet. Les descriptions sont bas�es sur les informations donn�es par le constructeur. Les descriptions sont dirig�es vers le type de donn�es re�ues ainsi que la pr�cision des capteurs et des moteurs. Ces descriptions sont importantes dans ce m�moire, car elles permettent de mieux connaitre les capteurs et les moteurs � notre disposition ce qui permet ainsi de mieux d�finir les erreurs qui sont propres � ces capteurs.

Le capteur infrarouge EV3 d�tecte la proximit� d'objets (jusqu'� 70 cm) et lit les signaux �mis par la balise infrarouge EV3 (distance max de 2m ). Les utilisateurs peuvent cr�er des robots t�l�command�s, faire des courses d'obstacles et se familiariser avec l'utilisation de la technologie infrarouge dont les t�l�commandes des TV, les syst�mes de surveillance.

Le capteur de couleur num�rique EV3 diff�rencie huit couleurs. Il peut �tre utilis� aussi comme capteur photosensible en mesurant l'intensit� lumineuse. Les utilisateurs peuvent construire des robots qui trient selon les couleurs ou suivent une ligne, exp�rimenter la r�flexion de la lumi�re de diff�rentes couleurs et se familiariser avec une technologie largement r�pandue dans les secteurs industriels du recyclage, de l'agriculture et de l'emballage. 

Le capteur tactile analogique EV3 est un outil simple, mais extr�mement pr�cis, capable de d�tecter toute pression ou rel�chement de son bouton frontal. Les utilisateurs pourront construire des syst�mes de commande marche/arr�t, cr�er des robots capables de s'extraire d'un labyrinthe et d�couvrir l'utilisation de cette technologie dans des appareils tels que les instruments de musique num�riques, les claviers d'ordinateur ou l'�lectrom�nager.

Le grand servomoteur EV3 est un puissant moteur avec retour tachym�trique pour un contr�le pr�cis au degr� pr�s. Gr�ce � son capteur de rotation int�gr�, ce moteur intelligent peut �tre synchronis� avec les autres moteurs d'un robot pour rouler en ligne droite � la m�me vitesse. Il peut �galement �tre utilis� pour fournir une mesure pr�cise pour des exp�riences. Par ailleurs, la forme du boitier facilite l'assemblage des trains d'engrenage.

Le servomoteur moyen EV3 est parfait pour des charges moins importantes, des applications � vitesse plus �lev�e, ainsi que des situations o� une plus grande r�activit� et un plus petit profil sont n�cessaires lors de la conception du robot. Le moteur se sert d'un retour tachym�trique pour un contr�le pr�cis au degr� pr�s et poss�de un capteur de rotation int�gr�.

\section{Description du software } 
\label{sec:DesSoftware}
C'est dans la brique EV3 qu'il est possible d'injecter un programme contenant les instructions du robot. Lego fournit un langage de programmation graphique qui s'appelle RoboLab(voir ~\ref{Robolab}). Il est le fruit de la collaboration de Lego avec le M.I.T.. Il se r�v�le un bon langage de programmation pour l'apprentissage de la programmation. Cependant, il n'est pas appropri� au d�veloppement d'algorithme complexe de localisation comme EKF ou bien MCL. Les briques Mindstorm ont vite �t� hack�es pour permettre de d�velopper des programmes � l'aide d'autre langage que RoboLab. Dans la version EV3, il est possible de booter sur une carte SD. Ce qui a permis que des OS tels que Linux soient adapt�s pour fonctionner sur la brique. Le projet ev3Dev est un exemple d'adaptation de Linux pour la brique EV3. Il est donc maintenant potentiellement possible de d�velopper � l'aide de tous les langages disponibles sur Linux. Lejos est une API �crite en Java comme le laisse supposer son nom. Lejos est l'outil principalement utilis� dans ce m�moire, une description plus compl�te est donc donn�e dans une section d�di�e.

\begin{figure}
\begin{center}
\includegraphics[scale=0.7]{./../img/Robolab.png}
\end{center}
\caption{Robolab}
\label{Robolab}
\end{figure}

\section{Lejos}
Lejos est un microprogramme libre destin� � remplacer le microprogramme originalement install� sur la brique Lego Mindstorm. Lejos inclut une machine virtuelle Java permettant de d�velopper des robots dans le langage Java. Ce qui a donn� le nom Lejos qui est le mot Legos o� le � g � a �t� remplac� par un � j � pour Java. Pour la version EV3 de la brique Lego, cette machine virtuelle Java est mise � disposition par Oracle. Lejos est disponible sur la brique RCX, NXT et EV3. Depuis son lancement jusqu'� aujourd'hui Lejos dispose d'une communaut� active et de taille assez importante. Le d�p�t officiel du code source est sourceforge.net. Entre le 01-01-2015 et le 09-04-2015 la version 0.9.0-beta de la brique EV3 a �t� t�l�charg�e 2700 fois. En effet, l'outil Lejos est couramment utilis� dans les universit�s et dans les hautes �coles pour l'apprentissage de la robotique et du langage Java. L'impl�mentation orient�e objet permet aux �tudiants d'�tudier la robotique avec le niveau d'abstraction qu'ils d�sirent. Il est ainsi possible de d�velopper des algorithmes de localisation sans se soucier des adresses hexad�cimales des moteurs et des capteurs. Des chercheurs de l'universit� de Porto au Portugal ont d�j� utilis� Lejos pour impl�menter un algorithme  qui cartographie d'un son environnement \cite{OliveiraLejos}. 
\subsection{MCL }
Lejos fournit dans sa librairie une impl�mentation de l'algorithme MCL. Dans cette impl�mentation les donn�es de mesures proviennent d'un capteur de distance.   

\chapter{Impl�mentation personnelle}
Cette section contient la description de mon impl�mentation de l'algorithme EKF dans la librairie Lejos. Par la suite, l'impl�mentation de l'algorithme EKF est compar�e � l'impl�mentation MCL d�j� pr�sente dans la librairie Lejos. 
\section{Description du robot construit}
La plateforme du robot est de type � differential wheeled robot �(voir ~\ref{DWR}). Ce qui consiste en deux moteurs ind�pendants positionn�s de fa�ons oppos�es sur le robot. Ce choix de plateforme est commun en robotique. Cette plateforme est simple � mettre en oeuvre et fournit une amplitude de mouvement importante. Les plateformes de type steering sont semblables aux voitures classiques. Elle demande un m�canisme plus complexe et leur amplitude de mouvement est moindre et donc moins adapt�e � la robotique.

\begin{figure}
\begin{center}
\includegraphics{./../img/DifferentialWheeledRobot.png}
\caption{Differential wheeled robot }
\source{\href{https://en.wikipedia.org/wiki/Differential_wheeled_robot}{Wikipedia},
 Auteur : Patrik}

\end{center}
\label{DWR}
\end{figure}




\section{Impl�mentation de l'algorithme EKF }

\subsection{D�tection de Feature avec la cam�ra du smartphone}
\label{sec:Detection de Feature avec la camera du smartphone}

Les codes QR sont des �l�ments faciles � identifier pour la cam�ra d'un smartphone. De nombreuses librairies de qualit� ont d�j� �t� d�velopp�es pour d�tecter et d�coder des codes QR. Zbar \footnote{ Zbar : \href{http://zbar.sourceforge.net/}{zbar.sourceforge.net}} est une de ces librairies open source. Elle est disponible sur Android et  IOS.  Pour ce m�moire, une application permettant d'estimer la distance du smarphone au code QR a �t� d�velopp�e � l'aide de la librairie Zbar. Pour pouvoir utiliser les codes QR pour estimer la distance entre eux et le smartphone, les codes QR doivent �tre d'une dimension donn�e (dans ce m�moire : un carr� de 10cm de cot� ). La librairie Zbar renvoie la dimension du code QR en nombre de pixels capt�s par la cam�ra. � l'aide d'un �talonnage de la cam�ra qui consiste � d�terminer l'ouverture de l'objectif et du calcul trigonom�trique suivant, il est possible de d�terminer la distance des codes QR de 10cm de cot�(voir ~\ref{EDQRC}).
$$Distance =  \frac{\frac{CapteurResolutionHorizontale }{MesureNombrePixelsHorinzontale} * LargeurCodeQR}  {2*\tan(\alpha)} $$


\begin{figure}
\begin{center}
\begin{tikzpicture}

    % define coordinates
    \coordinate (O) at (0,0) ;
    \coordinate (A) at (6,0) ;
    \coordinate (B) at (-2,0) ;

    \coordinate (E) at (3,3) ;
    \coordinate (F) at (3,-3) ;
    
   \coordinate (A_QR_R) at (3,1) ;
   \coordinate (B_QR_R) at (3,2) ;
   \coordinate (M_QR_R) at (3,1.5) ;
   
   



    % axis
    \draw[] (A) -- (B) ;
    

      \draw[blue] (E) -- (F) ;
      \draw[red,ultra thick] (A_QR_R) -- (B_QR_R) ;
      \node[right,red] at (3,1){Code QR de 10cm};
        \node[right,blue] at (3,-1){Vue globale du smartphone};
      
      \fill[black] (M_QR_R) circle (2pt);
      \draw[dash pattern=on5pt off3pt,ultra thick] (O) -- (M_QR_R) ;



    % rays
    \draw[dash pattern=on5pt off3pt,red,ultra thick] (O) -- (45:5);
    \draw[dash pattern=on5pt off3pt,red,ultra thick] (O) -- (-45:5);

    % angles
    \draw[ultra thick,red] (0.7,0) arc (0:45:0.7);
        \draw[ultra thick,black] (1.5,0) arc (0:26:1.5);
    \node[black] at (10:1.9)  {$\theta$};

    \node[red] at (10:0.9)  {$\alpha$};
    
    
\end{tikzpicture}
\end{center}
\caption{�valuation de la distance des codes QR}
\label{EDQRC}
\end{figure}

Il est �galement possible de d�terminer l'angle entre la direction du robot et le centre du code QR � l'aide de la formule suivante : 

$$\theta = \alpha-\frac{\alpha * 2*CentreCodeQRPixel}{ CapteurResolutionHorizontale} $$
si le code QR se trouve � droite du robot la formule devient :
$$\theta = \frac{\alpha * 2*CentreCodeQRPixel}{ CapteurResolutionHorizontale}-\alpha $$

L'�talonnage consiste � d�terminer $\alpha $ � l'aide de mesures faites � distance connue. 



\subsection{Les cartes }
La carte qui stocke les positions des codes QR  et qui est utilis�e par l'algorithme EKF est stock�e dans une image au format SVG. Ce format consiste � d�finir des �l�ments graphiques simples dans un fichier XML. Les codes QR sont donc repr�sent�s par une ligne de 10cm de longueur. La figure ~\ref{ekfmap} repr�sente cette carte o� les codes QR sont repr�sent�s en rouge et les murs en noir. 
Les murs sont �galement d�finis dans un fichier SVG diff�rent. Cette d�composition des cartes est volontaire et elle permet de charger uniquement les sous-cartes utiles � l'algorithme. Il n'est par exemple pas utile pour l'algorithme MCL d'avoir la carte compos�e des codes QR.  

En plus, de ces deux cartes qui ont �t� g�n�r�es � la main au pr�alable. Une carte dynamique de type grille d'occupation est g�n�r�e � l'aide du capteur infrarouge et du capteur de pression positionn�e � l'avant du robot. 


\begin{figure}
\begin{center}

\includegraphics[scale=0.7]{./../img/ekfmap.png}
\caption{carte EKF }
\label{ekfmap}
\end{center}
\end{figure}  

\subsection{Pseudo-code}

L'impl�mentation EKF utilise la technique de d�tection de codes QR comme features pr�sent�es dans la section ~\ref{sec:Detection de Feature avec la camera du smartphone}. 


Cette impl�mentation d'EKF compl�te la librairie Commons Math\footnote{Commons Math : \href{http://commons.apache.org/proper/commons-math/}{commons.apache.org/proper/commons-math}}  qui est une librairie math�matique open-source de Apache. Celle-ci contient une s�rie de classe permettant de manipuler et d'appliquer des op�rations sur des matrices. Ce qui se r�v�le tr�s utile dans les algorithmes de Kalman. Elle contient �galement une impl�mentation du filtre de Kalman, mais ne contient pas d'impl�mentation du Extended Kalman Filter. 

\subsection{Repr�sentation de la matrice de covariance}
La matrice de covariance est comme son nom l'indique une matrice qui permet de repr�senter la covariance entre chaque variable de la position du robot. Les valeurs contenues dans cette matrice sont tr�s utiles, mais restent fastidieuses � lire, car cette matrice change dynamiquement et r�guli�rement. Il a donc �t� important d'impl�menter une repr�sentation graphique des informations importantes de cette matrice de covariance. Cette repr�sentation permet de se faire une id�e rapide de ces valeurs, sans devoir les analyser une � une. La figure~\ref{cov} pr�sente la repr�sentation choisie de la covariance. La moyenne donne la position estim�e du robot. Cette position estim�e est repr�sent�e par le point bleu central qui correspond � la position $x,y$ estim�e du robot et la droite bleue au milieu des deux autres correspond � la direction estim�e du robot. L'ellipse autour de la position ainsi que les deux autres droites permettent de repr�senter la matrice de covariance. Voici la d�finition de la covariance pour mieux comprendre ce que repr�sente la covariance et comprendre comment sont construites cette ellipse et ces droites  : 

$$ Cov(X,Y) = E[(X-E[X])(Y-E[Y])]  $$  
o� $E[] $ d�signe l'esp�rance math�matique. La covariance caract�rise la variation simultan�e des deux variables al�atoires $X, Y$. Elle est positive lorsque la diff�rence entre les variables al�atoire $X,Y$ et leur moyenne ont tendance � �tre de m�me signe et n�gative dans le cas contraire. Soit le vecteur de position �crit :
$$\vec{X} = \begin{pmatrix} x \\ y \\ \theta \\ \end{pmatrix}$$

La matrice de covariance pour le vecteur de position est la suivante :

$$Var(\vec{X})= 
\begin{pmatrix} 
Var(x) & Cov(x,y)& Cov(x,\theta) \\ 
Cov(y,x)& Var(y) & Cov(y,\theta) \\ 
Cov(\theta,x) & Cov(\theta,y) & Var(\theta)\\
\end{pmatrix}
$$
La diagonale de la matrice de covariance est compos�e des variances des variables al�atoires de $\vec{X}$ ce qui est normal, car $Cov(X,X)= Var(X)$. La matrice de covariance est une matrice sym�trique, car $Cov(X,Y)=Cov(Y,X)$. Pour revenir � la repr�sentation de la matrice de covariance,  l'angle d'�cartement entre les deux droites de notre repr�sentation est donn� par $Var(\theta)$ ce qui caract�rise donc la dispersion des valeurs de la direction du robot. Plus cette variance est petite et plus la direction estim�e du robot est sure et inversement plus elle est grande et plus la direction est incertaine. L'ellipse est d�finie � l'aide de la sous-matrice suivante : 

$$
\begin{pmatrix} 
Var(x) & Cov(x,y)\\ 
Cov(y,x)& Var(y) \\ 
\end{pmatrix}
$$  
Cette technique\footnote{ Ellipse repr�sentation : \href{http://www.visiondummy.com/2014/04/draw-error-ellipse-representing-covariance-matrix/
}{www.visiondummy.com}} qui permet de visualiser la covariance d'une matrice � l'aide d'une ellipse peut �tre appliqu� � n'importe quelle matrice de covariance. 
$Var(x)$ et $Var(y)$ permettent de d�finir la largeur et la hauteur de l'ellipse � l'aide de l'�quation de l'ellipse suivante : 
$$
\left( \frac{x}{Var(x)} \right)^2 +\left ( \frac{y}{Var(y)}\right)^2 = 1 
$$

 Il faut maintenant d�terminer l'orientation de l'ellipse.  Lorsque 
$Cov(x,y) = 0$ l'orientation de l'ellipse est inchang�e. De fa�on g�n�rale l'angle d'orientation peut �tre d�fini par la formule suivante :  


$$
\alpha = arctan2 ( V_1.y,V_1.x )
$$

o� $V_1$ correspond au vecteur propre majeur et $\alpha$  correspond � l'angle entre $V_1$ et l'axe des x. Trouver le vecteur propre majeur consiste � r�soudre l'�quation suivante : 

  $$
  A\vec{v} = \lambda \vec(v)
  $$
o� $A$ correspond � la matrice de covariance, $v$ le vecteur propre et $\lambda$ la valeur propre. Cette �quation est r�solue � l'aide de la libraire Commons Math d�j� utilis�e pour manipuler les matrices de l'algorithme EKF. Cette �quation poss�de deux solutions. Le vecteur majeur correspond au vecteur qui poss�de la plus grande valeur propre. 


\begin{figure}
\begin{center}

\includegraphics[scale=0.7]{./../img/covariance.png}
\caption{Repr�sentation de la covariance }
\label{cov}
\end{center}
\end{figure}


\subsection{Tests et r�sultats de l'impl�mentation de EKF}





\begin{algorithm}
\caption{ EKFCodeQR  }\label{alg:EKFCodeQR }
\begin{algorithmic}[1]
\Procedure{EKFCodeQR }{$ \mu_{t-1}, \Sigma_{t-1},  u_t , z_t, m  $}  
\State $\theta \gets \mu_{t-1,\theta } $
\State $ G_t \gets 
\begin{pmatrix}
1&0& -d_t \sin( \theta)\\
0&1&d_t\cos(\theta)\\
0&0&1\\
\end{pmatrix}
$
\State $V_t \gets 
\begin{pmatrix}
\cos() & -d \sin\\
\sin()& d_t \cos \\
0&1\\
\end{pmatrix}
$
\State $M_t \gets 
\begin{pmatrix}
\sigma^2 &0\\
0&\sigma^2 \\ 
\end{pmatrix}
$

\State $\overline{\mu}_t \gets \mu{t-1} + 
\begin{pmatrix}
d \cos \\
d \sin \\
\gamma \\
\end{pmatrix}
$


\State $\overline{\Sigma}_t \gets G_t \Sigma_{t-1}G_t^T + V_tM_tV_t^T $
\State $ Q \gets 
\begin{pmatrix}
\sigma^2_r&0&0\\
0&\sigma^2_r&0\\
0&0&\sigma^2_r\\
\end{pmatrix}$

\ForAll{ observed features   {$ z^i_t \gets (d^i_t,\rho^i_t)^T $ }}
\State $q \gets (m_{j,x}-\overline{\mu}_{t,x} )^2 + (m_{j,y}-\overline{\mu}_{t,y})^2$
\State $ \hat{z}^i_t \gets 
\begin{pmatrix}
\sqrt{q}\\
atan2(m_{j,y}-\overline{\mu}_{t,y},m_{j,x}-\overline{\mu}_{t,x} )- \overline{\mu_{t,\theta}}\\
\end{pmatrix}
$
\State $H^i_t \gets
\begin{pmatrix}
-\frac{m_{j,x}-\overline{\mu}_{t,x}}{\sqrt{q}}     &    -\frac{m_{j,y}-\overline{\mu}_{t,y}}{\sqrt{q}}   &    0\\
\frac{m_{j,y}-\overline{\mu}_{t,y}}{q} & -\frac{m_{j,x}-\overline{\mu}_{t,x}}{q}            &  -1\\

\end{pmatrix}
$ 

\State $S^i_t \gets H^i_t \overline{\Sigma_t} [H^i_t]^T + Q_t $ 
\State $ K_t^i \gets \overline{\Sigma}_t [H_t^i]^T [S^i_t ]^{-1}$ \Comment{Kalman Gain}

\State $ \overline{\mu}_t \gets  \overline{\mu}_t  + K_t^i(z_t^i - \hat{z}^i_t ))  $ \Comment{mise � jour}
\State $ \overline{\Sigma}_t \gets (I - K_t^iH_t^i)\overline{\Sigma}_t$ \Comment{mise � jour}

\EndFor

\State $ \mu_t \gets  \overline{\mu}_t $  

\State $ \Sigma_t \gets \overline{\Sigma}_t $  

\State \textbf{return} $ \mu_t , \Sigma_t $
\EndProcedure
\end{algorithmic}
\end{algorithm}




\section{Comparaison avec le MCL}





\part{Conclusion}
\chapter{Conclusion}


Bien que les concepts globaux th�oriques des algorithmes semblent � premi�re vue simples, il s'av�re qu'impl�menter de tels algorithmes de localisation sur un robot r�el demande de r�soudre une quantit� importante de sous probl�mes.  En effet, un algorithme de localisation n'est pas un �l�ment isol�, il doit �tre incorpor� dans un tout coh�rent. Il est donc primordial d'avoir une compr�hension d'ensemble ainsi qu'une compr�hension d�taill�e de l'impl�mentation du robot. Il est �galement important de garder en t�te l'objectif du robot et d�terminer l'environnement dans lequel le robot �volue pour d�terminer l'algorithme qui correspond le mieux � ces caract�ristiques.     


\chapter{Recherches futures  }
Ce m�moire n'a fait qu'effleurer la localisation d'un robot mobile dans son environnement, ce qui n'est qu'un sous domaine de l'intelligence artificielle dans la robotique. Il y a donc un grand nombre de possibilit�s pour continuer ce m�moire. 

\section{Analyse d'images}
\label{sec:Analyse d'images}
Dans ce m�moire le choix d'utiliser des codes QR a permis de simplifier la d�tection de rep�re pour l'algorithme EKF. Cependant, il n'est pas possible d'utiliser des codes QR dans toutes les situations. Il est donc important de savoir extraire des rep�res des donn�es brutes des capteurs. De nombreux algorithmes permettent d'extraire des rep�re � l'aide d'une cam�ra. Ce domaine de recherche discute des diff�rents �l�ments qui sont consid�r�s comme de bons rep�res. Par exemple, la d�tection de bord se base sur de fortes modifications de la lumi�re dans une image, l'ensemble des points o� cette forte modification de lumi�re apparait est consid�r� comme un bord. L'algorithme de Canny est un exemple d'algorithme de d�tection de bords~\cite{Canny86acomputational}.De la m�me mani�re les algorithmes de d�tection de coins tentent de d�terminer les caract�ristiques d'un coin. L'algorithme de Moravec~\cite{Moravec_1980_22} et de Harris et Stephens~\cite{Harris88acombined} sont des exemples d'algorithmes de d�tection de coins.  



 


\section{Ajouter d'IA dans Lejos}


Dans un but p�dagogique, il serait int�ressant de continuer � d�velopper la librairie Lejos. Rappelons qu'elle est utilis�e par de nombreuses �coles et universit�s. Elle fournit d�j� un nombre important de classes pour manipuler un robot. Cet outil est donc parfait pour d�velopper en pratique des aspects th�oriques. Cependant, elle souffre du peu de classe permettant de d�velopper une intelligence artificielle. Il serait donc int�ressant d'impl�menter des algorithmes de localisation utilisant d'autres types de capteurs, ou bien des algorithmes de prise de d�cision pour atteindre un objectif. Ces algorithmes pourraient �tre assembl�s pour construire un tout coh�rent. De plus cette librairie est �crite en Java et avec un bon niveau de d�composition des �l�ments, elle est donc facilement r�utilisable dans d'autres projets. Elle pourrait ainsi devenir un outil semblable � ROS \footnote{ROS : \href{http://www.ros.org/}{www.ros.org} } ou MSRDS\footnote{Microsoft Robotics Developer Studio 4 : \href{https://www.microsoft.com/en-us/download/details.aspx?id=29081}{www.microsoft.com}} qui sont des outils professionnels de haute qualit� fonctionnant aussi bien sur des robots industriels que des robots de loisirs comme les drones.



% etc

% Si vous utilisez (conseillé) BibTeX pour votre bibliographie :
\bibliographystyle{acm}


\bibliography{memoire}% si le fichier BibTeX est memoire.bib

\end{document}
%%% Local Variables: 
%%% mode: latex
%%% TeX-master: t
%%% TeX-PDF-mode: t
%%% End: 
