


\section*{R�sum�}
L'objectif de ce m�moire est de d�velopper des aspects th�oriques de la robotique en y liant une validation pratique � l'aide de composants � faible cout. La limite de ces composants y est discut�e. La localisation d'un robot dans son environnement est le principal sujet abord�. Les algorithmes de localisation probabilistes, Extended Kalman Filter (EKF) ainsi que Monte Carlo Localization (MCL) seront �tudi�s en profondeur et compar�s. Toutefois, une multitude de sous-sujets sont li�s et d�coulent de la localisation d'un robot dans son environnement. Les principaux sous-sujets sont la construction de cartes de l'environnement du robot, la recherche de chemin entre un point de d�part et un point final, les syst�mes d'exploitation dans la robotique, la caract�risation des capteurs et des actuateurs, l'analyse d'images... Cet ensemble de sous-sujets est abord� dans ce m�moire pour permettre aux lecteurs de comprendre une partie de l'univers de la robotique et lui donner des pistes et des r�f�rences pour entamer des recherches futures. La compr�hension de l'univers de la robotique est primordiale pour int�grer les algorithmes de localisation en un tout coh�rent. 

Le kit EV3 de Lego est utilis� pour la validation pratique des concepts th�oriques pr�sent�s. Ce kit est compos� d'une brique intelligente, de capteurs, de moteurs ainsi que d'�l�ments de construction qui permettent de r�aliser rapidement la structure d'un robot. Ce kit est combin� avec un smartphone tournant sous Android. L'ajout du smartphone permet de se servir de son puissant processeur et de ses capteurs. En effet, les smartphones actuels poss�dent un grand nombre de capteurs � couts r�duits et leur capacit� de calcul est importante. Le principal capteur utilis� est la cam�ra du smartphone. 
